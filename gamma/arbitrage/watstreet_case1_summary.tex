\documentclass{article}
% % Required for inserting images
% % \documentclass[12pt]{article}

% \usepackage{fullpage,amssymb}
% \usepackage{amsthm}
% \documentclass{article}

\begin{document}

\title{\textbf{Summary of Case Study: Crypto Arbitrage Opportunity Between Poloniex and Binance}}
\maketitle

Please add an author list as well!

This outline is just a suggested template, feel free to add/change up the content, but ensure you include at least the 6 main sections: Abstract, Problem, Analysis, and Proposed Strategy, Conclusions, References. Feel free to create your own graphics, change up the document style (as long as it looks somewhat professional/presentable). Ideally, it should be 1-3 pages summarizing your presentation and work for people who can't just be there at your presentation or would like to just read about what you've done. This would possibly be shown to potential clients and shouldn't be too short/long either.
\section{Abstract}
[Insert a 4-5 line paragraph here explaining everything about this case study as a short summary. Things to include: the problem/case, your findings + conclusion. Think like an elevator pitch.]

\section{Problem}
\subsection{Intro}
Feel free to copy the intro of the original case study or give a quick few sentences about the problem you tackled in this case study.
\subsection{Data}
Talk about the available data and what possible methodologies you used/thought of to approach the problem. Doesn't have to be long.

\section{Analysis}
This section will mostly be used to summarize your findings, maybe add any thing you researched, the models you created to do the analysis, and what you concluded about the different parts of the analysis. Feel free to add more subsections if you did any extra research/analysis.
\subsection{Indicators}
In order to compare arbitrage opportunities accross exchanges we will need a time series of bid/ask data for both exchanges. What would be best is a dataset that contains all orderbook updates for both exchanges. With this we can compare the bid/ask prices of the two exchanges to determine if arbitrage opportunities are possible. We will also have to include trading fees associated with each exchange.
\subsection{Duration}
Insert.
\subsection{Driving Factors}
- market liquidity: leads to pricing discrepancies 
- regulatory differences: different countries/jurisdictions may have different policies/restrictions that affects the pricing and availabilities for currencies and assets 
- geographical factors: may leads to regulatory differences; also includes time zone differences for example, leading to price disparities during overlappign trading hours 
- technological advancements: technonlogy often increases market efficiency and reduces arbitrage duration 
- market inefficiencies: arbitrage arises when where are some temporary discrepancies in pricing caused by market inefficiency
\subsection{Additional Exchanges}
- Coinbase Pro, also known as GDAX (Global Digital Asset Exchange)
- Bitstamp
- KuCoin
- OKEx
\subsection{Illiquidity}
- Order Book Depth: A deeper order book indicates a higher number of buy and sell orders at various price levels, which generally leads to better liquidity
- Trading Volume: Higher trading volumes suggest greater market activity and ease of executing trades
\subsection{Limitations}
In this subsection, explain any limitations of the data or resources or information and any constraints on your analysis.

\section{Proposed Strategy}
Insert a quick two lines summarizing your suggested approach. Then in each subsection, explain/expand on the why and ensure your reader knows that your strategy is profitable if it is. Consider talking about the later sections in the original case studies.
\subsection{Building Positions}
Insert.
\subsection{Transferring to Binance}
Insert.
\subsection{Profitability}
Insert.
\subsection{Risks}
Insert.

\section{Conclusion}
Maybe insert overall conclusion / suggestions. Add any comments you have.

\section{References}
This is tedious but necessary. Please add references in appropriate format (MLA).
\end{document}
