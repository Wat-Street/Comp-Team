\documentclass{article}

\begin{document}

\title{\textbf{Summary of Case Study: Crypto Arbitrage Opportunity Between Poloniex and Binance}}
\maketitle

\section{Abstract}
[Insert a 4-5 line paragraph here explaining everything about this case study as a short summary. Things to include: the problem/case, your findings + conclusion. Think like an elevator pitch.]

\section{Problem}
\subsection{Intro}
This case study focuses on identifying and capitalizing on a crypto arbitrage opportunity between two exchanges, Poloniex and Binance. We will examine various currency pairs, including DOT/USDT, LTC/USDT, and TRX/USDT. The objective is to analyze indicators that signify arbitrage opportunities, determine their
duration, and understand the primary driving factors behind them. Additionally, we will explore how to
leverage illiquidity and strategically build positions on Poloniex for subsequent transfer to Binance. Finally,
we will consider the required inventory amount to break even on withdrawals.
\subsection{Data}
The data used in this case study is from the Poloniex and Binance exchanges in the form of tick-by-tick trades. The data was collected from a complete day of trading on August 15, 2022 for the DOT/USDT, LTC/USDT, and TRX/USDT currency pairs. These currency pairs were chosen because they are all traded on both exchanges and have a high trading volume. An additional reason for choosing these currency pairs is that they are all traded against USDT, which is a stablecoin that is pegged to the US dollar. This means that we can assume that the price of USDT is constant and that any price differences between the exchanges are due to arbitrage opportunities.

\section{Analysis}
[This section will mostly be used to summarize your findings, maybe add any thing you researched, the models you created to do the analysis, and what you concluded about the different parts of the analysis. Feel free to add more subsections if you did any extra research/analysis.]
\subsection{Indicators}
In order to compare arbitrage opportunities accross exchanges we will need a time series of bid/ask data for both exchanges. What would be best is a dataset that contains all orderbook updates for both exchanges. With this we can compare the bid/ask prices of the two exchanges to determine if arbitrage opportunities are possible. We will also have to include trading fees associated with each exchange.
\subsection{Duration}
[By closely monitoring price differentials over time, we can determine the average duration of arbitrage
opportunities. This analysis will help us estimate the window of time available for executing profitable
trades.]
\subsection{Driving Factors}
- market liquidity: leads to pricing discrepancies
- regulatory differences: different countries/jurisdictions may have different policies/restrictions that affects the pricing and availabilities for currencies and assets
- geographical factors: may leads to regulatory differences; also includes time zone differences for example, leading to price disparities during overlappign trading hours
- technological advancements: technonlogy often increases market efficiency and reduces arbitrage duration
- market inefficiencies: arbitrage arises when where are some temporary discrepancies in pricing caused by market inefficiency
\subsection{Additional Exchanges}
- Coinbase Pro, also known as GDAX (Global Digital Asset Exchange)
- Bitstamp
- KuCoin
- OKEx
\subsection{Illiquidity}
- Order Book Depth: A deeper order book indicates a higher number of buy and sell orders at various price levels, which generally leads to better liquidity
- Trading Volume: Higher trading volumes suggest greater market activity and ease of executing trades
\subsection{Limitations}
[In this subsection, explain any limitations of the data or resources or information and any constraints on your analysis.]

\section{Proposed Strategy}
[Insert a quick two lines summarizing your suggested approach. Then in each subsection, explain/expand on the why and ensure your reader knows that your strategy is profitable if it is. Consider talking about the later sections in the original case studies.]
\subsection{Building Positions}
[To strategically build positions on Poloniex, we will focus on the currency pairs with the most significant price differentials. By executing buy orders on Poloniex at lower prices and simultaneously placing sell orders on Binance at higher prices, we can capitalize on the price discrepancy.]
\subsection{Transferring to Binance}
[When transferring assets from Poloniex to Binance, we need to account for withdrawal fees and minimum
withdrawal amounts. Calculating the break-even point for withdrawals will enable us to determine the
required inventory amount on Poloniex.]
\subsection{Profitability}
Insert.
\subsection{Risks}
Insert.

\section{Conclusion}
This case study aimed to identify and exploit crypto arbitrage opportunities between Poloniex and Binance. Through the analysis of indicators, such as price differentials, duration of opportunities, and driving factors, we can gain insights into the dynamics of arbitrage with crypto. We also explored how to leverage illiquidity and strategically build positions on Poloniex for subsequent transfer to Binance. Finally, we considered the required inventory amount to break even on withdrawals. Overall, we found that arbitrage opportunities are common and can be exploited with the right strategy. However, there are many risks involved, such as regulatory and technological factors, that can lead to losses. Therefore, it is important to be aware of these risks and to have a well thought out strategy before engaging in arbitrage.

\section{References}

\end{document}
